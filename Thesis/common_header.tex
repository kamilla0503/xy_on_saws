%
% A COMMON HEADER FOR THE SLIDES
%

\usepackage{amsmath,amssymb,amsthm}
\usepackage{mathtools}
\usepackage{xcolor}
\usepackage{accents}

\usepackage{verbatim}

%%% Use a HSE mandated font, Myriad Pro
\usefonttheme{professionalfonts}
\usepackage[T1]{fontenc}
\usepackage{mathspec}
%\usepackage{sfmath}
\setmainfont{Myriad Pro}
\setsansfont{Myriad Pro}
\setmathfont{Asana Math}

%%% Use a plain page style
\usetheme{default}

%%% Remove navigation symbols from bottom right
%%% https://tex.stackexchange.com/questions/686/how-to-get-rid-of-navigation-symbols-in-beamer
\beamertemplatenavigationsymbolsempty


%%% Add an HSE-mandated backgound : presentation, yes; tablet, no.
%%% https://tex.stackexchange.com/questions/7916/how-to-insert-a-background-image-in-a-beamer-frame
%\usebackgroundtemplate%
{%
%    \includegraphics[width=\paperwidth,height=\paperheight]{images/BACKGROUND.jpg}%
}

% set the path to the images folder: historically they are in ./images/
% and \includegraphics{...} uses an explicit images/... folder
% Now that the TeX sources are in subfolders, we need to append the graphicspath
\graphicspath{{../}}


%%%% Add slide numbers at bottom right
\setbeamertemplate{sidebar right}{}
\setbeamertemplate{footline}{%
\hfill\usebeamertemplate***{navigation symbols}
\hspace{0.3cm}\insertframenumber{}/\inserttotalframenumber}


%%%%%%%%%%%%%%%%%%%%%%%%%%%%%%%%%%%%%%%%%%%%%
\newcommand{\rmd}{\mathrm{d}}
\newcommand{\rme}{\mathrm{e}}
\newcommand{\vv}[1]{\mathbf{#1}}
\newcommand{\op}[1]{\widehat{#1}}
\DeclareMathOperator{\rot}{rot}
\DeclareMathOperator{\curl}{curl}
\DeclareMathOperator{\divr}{div}
\DeclareMathOperator{\grad}{grad}
\DeclareMathOperator{\Tr}{tr}
\DeclareMathOperator{\rank}{rank}
\DeclareMathOperator{\diag}{diag}
\DeclareMathOperator{\ran}{ran}
\DeclareMathOperator{\nnull}{null}
\DeclareMathOperator{\nspan}{span}
\DeclareMathOperator{\cond}{cond}
\DeclareMathOperator{\sign}{sign}
\DeclareMathOperator{\Real}{Re}


\newcommand{\la}{\langle}
\newcommand{\ra}{\rangle}
\newcommand{\bra}[1]{\la{#1}|}
\newcommand{\ket}[1]{|{#1}\ra}
\newcommand{\brak}[2]{{\la{#1} \cdot {#2}\ra}}

\newcommand{\vhatx}{\widehat{\vv{x}}}
\newcommand{\vx}{\vv{x}}
\newcommand{\vy}{\vv{y}}
\newcommand{\vb}{\vv{b}}
\newcommand{\vr}{\vv{r}}
\newcommand{\ve}{\vv{e}}
\newcommand{\va}{\vv{a}}
\newcommand{\vA}{\vv{A}}
\newcommand{\vB}{\vv{B}}
\newcommand{\vD}{\vv{D}}
\newcommand{\vL}{\vv{L}}
\newcommand{\vU}{\vv{U}}
\newcommand{\vP}{\vv{P}}
\newcommand{\vQ}{\vv{Q}}
\newcommand{\vV}{\vv{V}}
\newcommand{\vT}{\vv{T}}
\newcommand{\vu}{\vv{u}}
\newcommand{\vF}{\vv{F}}
\newcommand{\vR}{\vv{R}}

\newtheorem*{def*}{Def:}
\newtheorem*{theorem*}{Theorem:}
\newtheorem*{question*}{Question:}

%https://tex.stackexchange.com/questions/1874/how-to-make-a-larger-dot
\newcommand*{\dt}[1]{%
   \accentset{\mbox{\large\bfseries .}}{#1}}
\newcommand*{\ddt}[1]{%
%   \accentset{\mbox{\large\bfseries .\hspace{-0.25ex}.}}{#1}}
   \accentset{\mbox{\large\bfseries ..}}{#1}}

\newcommand{\Q}{{\Large\textbf{\textcolor{blue}{Q:~}}}}
\newcommand{\A}{{\Large\textbf{\textcolor{blue}{A:~}}}}

\definecolor{rred}{HTML}{cb4154}
\newcommand{\rred}[1]{{\color{rred} #1 }}

\newcommand{\red}[1]{{\color{red} #1 }}
\newcommand{\blue}[1]{{\color{blue} #1 }}
\newcommand{\green}[1]{{\color{OliveGreen} #1 }}

\newcommand{\rx}{\rred{x}}


%%%%%%%%%% for long text under the underbraces
\newcommand{\underbracewithoutspace}[2]{\mathrlap{\underbrace{\phantom{#1\strut}}_{#2}}#1}
\newcommand{\undertab}[1]{\rlap{%
  \begin{tabular}{l@{}}#1\end{tabular}}}
%%%%%%%%%%%usage:
%
%  \underbracewithoutspace{a+b}{\undertab{a long blah blah blah}}+c
%

\newcommand{\splittwo}[2]{\displaystyle \text{#1} \makebox[0pt][l]{#2} }
%
%  \underbracewithoutspace{a+b}{ \splittwo{a long}{blah blah blah} }
%
%    this is almost OK, but still leaves a gap above the brace if the 1st arg of \splittwo is too long

%%%%%%%%%%%%%%%%%%%%%%%%%%%%%%%%%%%%%%%%%%%%%

