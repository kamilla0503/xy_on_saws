%%% Results %%%
\chapter{Results} \label{ch:results}
To perform MC simulations for short chains from $N=100$ to $N=1000$, we run at least $2.1 \times 10^9$ MC steps using two  types of updates: snake-like and reconnect. For longer chains $N>1000$ we additionally use cluster update. For $N=4900$, we run at least $8 \times 10^{10} $ MC steps. 
\section{XY model on SAWs, 2D}
\subsection{Tests for validation simulations}
To test our Monte-Carlo (M) simulation, we compare results obtained using MC and Sampling + Exact Enumeration (EE). This method is not reliable, but it helps to make approximate checks. For short chains ($N=5, N=8$), we generate all set of self-avoiding walks by EE and sample spin configurations by uniform distribution $U(-\pi, \pi)$. As this is resource-consuming procedure, we sample spin configurations only $600$ times (so, 600 sequences of spins applied to each conformation) and repeat this $10$ times. Figure \ref{fig:ee} shows obtained results.  For $J=0$, the second moment of magnetization are close to exact values \eqref{m2j0} and the mean energy starts at $\langle e \rangle = 0$.

 \begin{figure}[H]
	\centering
	\includegraphics[scale=0.26]{Images/EE.png}
	\caption{$h=0$. Mean Radius \eqref{endtoend}, mean energy \eqref{hamiltonian} and   second moment of magnetization \eqref{secondmomentmagnetization}.   }
	\label{fig:ee}
\end{figure}


\subsection{Thermodynamic properties}
First, we measure the mean square magnetization \eqref{secondmomentmagnetization} and the mean energy \eqref{hamiltonian} for short chains (up to $N=1000$) in the large range of the interaction energy $J$ and for long chains (up to $N=4900$) in the narrow range where the system are expected to undergo the phase transition.

 \begin{figure}[t]
	\centering
	\includegraphics[scale=0.23]{Images/energy_shortchains.png}
	\includegraphics[scale=0.23]{Images/magnetization2_shortchains.png} \\
	\includegraphics[scale=0.23]{Images/energy_longchains.png}
	\includegraphics[scale=0.23]{Images/magnetization2_longchains.png}
	\caption{$h=0$. Mean energy \eqref{hamiltonian} and   second moment of magnetization \eqref{secondmomentmagnetization}. }
	\label{fig:energymagshort}
\end{figure}

 Figure \ref{fig:energymagshort} (left column) shows computational results for the mean energy  \eqref{hamiltonian} as a function of $J$. At the top plot for short chains, the mean energy starts at $\langle e \rangle = 0$ as expected for unfolded disordered SAWs. As $N \rightarrow \infty$, the value of the mean energy decreases and goes to the asymptotic value $\langle e \rangle = -2J$ for compact ordered walk. 
 
  Figure \ref{fig:energymagshort} (right column) illustrates obtained numerical results for the second moment of magnetization \eqref{secondmomentmagnetization}. At $J=0$, results are consisted with the exact solution \eqref{m2j0} and $  \langle m^2 \rangle  \rightarrow 0$ as $N \rightarrow \infty$. As $J$ increases, the square of magnetization grows up. One can suppose an ordering behavior for large J.
  
  From both energy and the second magnetization moment, we can clearly see the finite size effect. The longer chains has jumps in the function for lower $J$ in comparison to shorter ones. 
 
 
%Figure \ref{fig:energymagshort} illustrates obtained results for Mean energy \eqref{hamiltonian} and   second moment of magnetization \eqref{secondmomentmagnetization}. The system tends to order as interaction energy $J$ gets larger. 
\subsection{Structural properties}

Next, we estimate critical exponent $\nu$ \eqref{r_scale} from the asymptotic power law for the the mean square end-to-end distance of SAWs. We use following ansatz  \cite{Berretti1985}:
\begin{equation}
\label{berettiscale}
\log (R_N^2+k_1 ) = 2 \nu \log (N+k_2) + b.
\end{equation}
Here $k_1=k_2=1$ are phenomenological parameters. 
 \begin{figure}[H]
	\centering
	\includegraphics[scale=0.36]{Images/nu_shortchains.png}
	\caption{$h=0$. Estimations with errorbars of critical exponent $\nu$ .   }
	\label{fig:nushort}
\end{figure}
 

For the start, we perform curve-fitting for short chains on the large range of values $J$. Figure \ref{fig:nushort} illustrates obtained results of exponent estimation. In $J=0$, the critical exponent $\nu$ equals $\nu = \frac{3}{4}$  which is consisted with the value of non-interacting SAWs \eqref{nur}. The value $\nu=\frac{4}{7}$, which is the exact value
for interacting SAWs \eqref{nu_theta}, appears at the region $ 1.25 < J < 1.4$. In previous section we showed that the energy and second magnetization moment functions of $J$ have jumps approximately at the same region. 

  We can assume that XY model on SAWs also has value $\nu = 4/7$ \eqref{nu_theta} at the point of structural phase transition. We use this value to obtain collapsing plots in Figure \ref{fig:bcshort} in following subsection \ref{section:Transition}. 

%\section{Scalings}
%Next, we move to estimation of the crossover exponent $\phi$ which
%quantifies the deviation from criticality via the scaled coupling $x = (J-J_c) / N^{-\phi}$ \cite{van2015statistical}. In \cite{PhysRevE.104.024122, PhysRevE.104.054501} it was assumed the value $\phi \approx 0.71$. We use this value to produce scaling plots \ref{fig:radiusscaling}. 
%\begin{figure}[H]
%	\centering
%	\includegraphics[scale=0.23]{Images/R2_data_collapse_phi.png} 	\includegraphics[scale=0.23]{Images/R2_data_collapse_phi1.png} \\ 
%	\includegraphics[scale=0.23]{Images/rscalinglong_dc.png}
%	\includegraphics[scale=0.23]{Images/rscalinglong_dc1.png} 	\\
%		\includegraphics[scale=0.23]{Images/2dmag2scaling.png} 	\includegraphics[scale=0.23]{Images/2dmag2scaling1.png}  
%	\caption{$h=0$. Mean radius \eqref{endtoend} and second moment of magnetization \eqref{secondmomentmagnetization} scalings. For left plots we use value $J_{\theta}=1.294$ and for right plots $J_{theta}=1.307$. We assume $\phi=5/7$. }
%	\label{fig:radiusscaling}
%\end{figure}






\subsection{Transition} \label{section:Transition}

 \begin{figure}[H]
 	\centering
 	\includegraphics[scale=0.23]{Images/bindercumulants_shortchains.png} 	\includegraphics[scale=0.23]{Images/bindercumulants_longchains.png} \\ 
 	 	\includegraphics[scale=0.23]{Images/rscaling_shortchains.png}
 	\includegraphics[scale=0.23]{Images/rscaling_longchains.png} 	
 	
 	\caption{$h=0$. Binder  cumulants \eqref{binderqum} and mean radius \eqref{endtoend}.  }
 	\label{fig:bcshort}
 \end{figure}


\subsection{Distribution of $\langle cos \theta \rangle$ and $\langle e \rangle$ }
  \begin{figure}[H]
 	\centering
 	\includegraphics[scale=0.25]{Images/distr_cos_1600.png}
 	\includegraphics[scale=0.25]{Images/distr_energy_1600.png} \\
 	\includegraphics[scale=0.25]{Images/distr_cos_2500.png}
 	\includegraphics[scale=0.25]{Images/distr_energy_2500.png}
 	\\
 	\includegraphics[scale=0.25]{Images/distr_cos_4900.png}
 	\includegraphics[scale=0.25]{Images/distr_energy_4900.png}
 	\caption{$h=0$.  }
 	\label{fig:distributions}
 \end{figure}

 \begin{figure}[H]
	\centering
	\includegraphics[scale=0.36]{Images/distr_energy_4900_time.png}
	\caption{$h=0$.  }
	\label{fig:distributione4900}
\end{figure}

 \begin{figure}[H]
	\centering
	\includegraphics[scale=0.36]{Images/distr_cos_4900_J.png}
	\caption{$h=0$.  }
	\label{fig:distributioncos4900}
\end{figure}